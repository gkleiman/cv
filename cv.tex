\documentclass[11pt,a4paper]{moderncv}

% moderncv themes
% optional argument are 'blue' (default), 'orange', 'red', 'green', 'grey' and
% 'roman' (for roman fonts, instead of sans serif fonts)
\moderncvtheme[blue]{classic}

\setlength{\hintscolumnwidth}{0.17\textwidth}
\recomputelengths

% for web links
\definecolor{href}{rgb}{0.2,0.4,0.65}
\newcommand\weblink[2] {{\color{href} \href{#1}{#2}}}
\newcommand\cvskill[2] {\cvline{#1}{
\begin{minipage}[t]{%
  \linewidth}\small #2
\end{minipage}
}}

% character encoding
\usepackage[utf8]{inputenc}

% adjust the page margins
\usepackage[scale=0.85]{geometry}
% if you want to change the width of the column with the dates
%\setlength{\hintscolumnwidth}{3cm}
% only for the classic theme, if you want to change the width of your name
% placeholder (to leave more space for your address details)
%\AtBeginDocument{\setlength{\maketitlenamewidth}{6cm}}
% required when changes are made to page layout lengths
\AtBeginDocument{\recomputelengths}

% personal data
\firstname{Gastón}
\familyname{Kleiman}
\title{Software Engineer}
\address{Virrey Arredondo 2631 5A}{C1426DZI Capital Federal, Argentina}
\mobile{+54 911 6370 1275}
\email{gaston.kleiman@gmail.com}
\homepage{github.com/gkleiman}
\extrainfo{Nationality: Argentine and German Citizenship}
% \quote{Some quote (optional)}

% uncomment to suppress automatic page numbering for CVs longer than one page

\nopagenumbers{}

\begin{document}
\maketitle
\vspace*{-2ex}
\section{Education}
\cventry{2003 -- 2011}{Informatics Engineering (6-year degree)}{Universidad de
Buenos Aires}{Argentina}{}{Final Project:
\weblink{https://github.com/gkleiman/wide}{wIDE} (Web Integrated Development
Environment).}
\cventry{2008 -- 2009\\ Sep \ \ \ \ \ Feb}{Research Project at the
\weblink{http://www.ibr.cs.tu-bs.de/dus/index.html}{Institute of Distributed
and Ubiquitous Systems}}{Technische Universität
Braunschweig}{Germany}{}{Participated in a research project on collaborative
transmission in wireless sensor networks using feedback based distributed
adaptive beamforming.\\Simulated and evaluated different synchronization and
clustering approaches using MATLAB.}
\cventry{2001\\$4^{1/2}$ months}{Study Abroad}{Gymnasium
Kaiser-Friedrich-Ufer}{Hamburg}{Germany}{Lived with a German family and
attended school.}

\section{Languages}
\cvlanguage{English}{Fluent}{First Certificate Exam (passed with grade A),
Cambridge, 2001}
\cvlanguage{Spanish}{Native}{}
\cvlanguage{German}{Fluent}{Deutsches Sprachdiplom der KMK Stufe II, 2002}

\section{Professional Experience}
\cvline{Jul 2012 -- current}{\textbf{Software Engineer (Contractor)},
\emph{\weblink{http://linkedin.com}{LinkedIn}}}
%
\cventry{2012 -- 2012\\ May -- Jul}{Android Developer}
{\weblink{http://etermax.com/}{Etermax S.A.}}{Buenos Aires, Argentina}{}{%
  Short gig developing a game similar to
  \weblink{https://play.google.com/store/apps/details?id=com.zynga.scramble}{Zynga's Scramble with Friends}
}
%
\cventry{2011 -- 2012\\ Aug -- Apr}{Business Intelligence \& Data
Warehouses - Corporate Engineering}
{\weblink{http://www.google.com}{Google} (Vendor),
\weblink{http://www.globant.com}{Globant}}{Buenos Aires, Argentina}{}{%
\begin{itemize}
  \item Develop ETL pipelines in Java for large scale datasets using
proprietary distributed computing libraries (MapReduce) deployed on Google's
cloud infrastructure.
  \item Implement/Maintain several executive dashboards using proprietary
JavaScript/AJAX data visualization libraries and Data Warehouses.
  \item Maintain a real-time Python/Java Business Intelligence platform to
monitor credit exposure and revenue risk.
  \item Release management for the Financial Datawarehouse Pipelines.
\end{itemize}
}
%
\cvline{1999 -- current}{\textbf{Freelance Independent Software Engineer}.
Working in projects such as small websites development.}
%
\cventry{2009 -- 2010\\ Oct \ \ \ \ \ Jun}{Software Engineer}
{\weblink{http://www.3dgames.com.ar}{3DGames}}{Buenos Aires, Argentina}{}{%
\begin{itemize}
  \item Migrated the web site of \weblink{http://club.speedy.com.ar}{Club
    Speedy} (Membership Reward Program of one of the largest ISPs in Argentina)
    from various PHP applications to a fully integrated site developed with
    Ruby on Rails. Responsible of meetings with the client and of designing,
    implementing and deploying the web site.
  \item Participated in the implementation of a Single Sign--On solution for
    the 3DG Platform using the Central Authentication Service
    (\weblink{http://en.wikipedia.org/wiki/Central_Authentication_Service}{CAS})
    protocol. Tools used:
    \weblink{https://github.com/relevance/castronaut}{castronaut},
    \weblink{https://wiki.jasig.org/display/CASC/phpCAS}{phpCAS}.
  \item Installed and configured multiple GNU/Linux servers. Responsible for
    emergency response in case of failures.
  \item Participated in the hiring process interviewing candidates.
\end{itemize}
}
%
\cventry{2010}{Undergrad Teaching Assistant}
{Universidad de Buenos Aires}{Buenos Aires, Argentina}{}{%
\begin{itemize}%
  \item ``Algorithms and Programming I'' and ``Algorithms and Programming II''.
\end{itemize}
}
%
\cventry{2009\\ Jan -- Mar}{Research Assistant}
{TU Braunschweig}{Braunschweig, Germany}{}{%
Wrote an Installation and Basic Usage Guide for the Universal Software Radio
Peripheral (\weblink{http://www.ettus.com/products}{USRP\texttrademark}) and
GNU Radio.
}
%
\cventry{2007 -- 2008\\ Jul \ \ \ \ \ Aug}{Java Developer / Consultant}
{\weblink{http://www.newtechnologies.com.ar}{New Technologies S.R.L.}}{Buenos
Aires, Argentina}{}{%
Participated in a team developing Java applications, and
worked as an internal and external consultant.
\begin{itemize}
  \item Developed a workflow based web application to track and optimize the
    Procurement Process of the \weblink{http://www.telefonica.com/}{``Grupo
    Telefónica''} in Argentina.  Tools used: jBPM, Tapestry, Spring, Hibernate,
    HttpUnit (for web scraping).
  \item Implemented prototype applications on J2EE in order to evaluate the
    adoption of different technologies (JavaServer Faces and Mule).
\end{itemize}
}

% \section{Profile} Versatile \& pragmatic developer with strong theoretical
% knowledge and practical experience ranging from system programming in
% low--level languages to graphic and web applications in high--level
% languages. Interested in developing for mobile platforms (specially for
% Android).

% Proactive, self--taught, quick--learner, goal oriented. Able to work
% effectively both as part of small/big teams as well as alone on own
% motivation.  Deep understanding and use of FLOSS (Free Libre Open Source
% Software) development methodologies and tools. Used to working under pressure
% and meeting demanding deadlines.

\section{Technical Skills}
\cvskill{Proficient in:}{Java, Python, GNU/Linux.}
\cvskill{Acquainted with:}{Ruby, C, C++, JavaScript, SQL, HTML, CSS.}
\cvline{Worked with:}{%
  \begin{minipage}[t]{\linewidth}\small
    \begin{itemize}
      \item Google Proprietary Technology.
      \item Android SDK.
      \item Ruby On Rails, RSpec, jQuery, jQuery UI.
      \item Perforce, git, darcs, svn, Mercurial.
      \item Bash, Eclipse, Vim, JUnit, Apache 2, Apache Tomcat, MySQL.
    \end{itemize}
  \end{minipage}
}
\cvskill{OOP}{Experienced in Object--oriented design and programming. Knowledge
of Design Patterns and UML.}
% \begin{itemize}
% \item Deep web development knowledge, ranging from back--end programming in
% Ruby, Java and PHP, to front--end programming using JavaScript and HTML + CSS.
% Knowledge of back--end frameworks (Rails, Tapestry) and front--end libraries
% (jQuery, jQuery UI). Prefer back--end rather than front--end programming.
% \item Object--oriented design and programming.
% \item Strong practical knowledge of Java, Ruby, C++ and C.
% \item Experience teaching programming with Python and using it to make small
% programs and utilities.
% \item Concurrent and network programming, including POSIX threads, BSD sockets,
% TCP/IP and POSIX IPC (Inter--Process Communication) experience.
% \item GNU/Linux systems specialist. More than 10 years of experience as user
% and administrator. Knowledge about its architecture, best--practices, shell
% scripting,  and web (Apache 2, Tomcat) \& database (MySQL) servers
% administration.
% \item Practical and theoretical knowledge of networking protocols (IP, TCP/IP,
% UDP, etc.) and application protocols over IP (HTTP, SMTP, POP 3, DNS, etc.).
% \item Network administration: firewalling (iptables), advanced routing, NAT.
% \item Basic knowledge of the Android SDK and mobile programming.
% \end{itemize}

\section{Contributions to FLOSS Projects}
\cvline{}{Contributed to popular FLOSS (Open--Source) projects including:}
\cvlistdoubleitem{\mbox{\weblink{http://mercurial.selenic.com/}{Mercurial}}}{%
\mbox{\weblink{http://ace.ajax.org/}{ACE (Ajax.org Text Editor)}}}
\cvlistitem{\mbox{\weblink{http://github.com/fesplugas/typus}{Typus}}}

\section{Personal Projects}
\cvskill{\mbox{\weblink{https://github.com/gkleiman/wide}{wIDE}}\\2010 --
current}{wIDE $[$Rails 3, jQuery, jQuery UI, delayed\_jobs, Mercurial, git$]$
is a web IDE that runs in the browser, but lives in a server. It was developed
for the \weblink{http://labi.fi.uba.ar/club_robotica.html}{Robotics Club} of
the Faculty of Engineering of the UBA, and allows them to edit and compile
programs for different microcontroller architectures from anywhere, anytime. It
allows the versioning of the projects using Mercurial.}
\cvskill{\mbox{\weblink{http://yanf.sourceforge.net/}{yanf}}\\2001}{Yanf
$[$C$]$ is a deprecated news fetcher coded by me, when I was 16 and web RSS
readers were not popular. It was published under the GPL v2 license, but is no
longer functional and is still available only for nostalgic reasons.}

\section{Awards}
\cventry{2008}{ALEARG Scholarship}{}{granted by the German Academic Exchange
Service (DAAD) to do research work at the TU Braunschweig}{Germany}{}{}{}{}
\cventry{2001}{Argentinian National Olympiad of Chemistry (OAQ)}{Bronze medal}{}{}{}{}

\section{Hobbies}
\cvlistitem{Cinema, Photography and Cooking.}
\cvlistitem{Tweaking and tunning my development environment, constantly
discovering new features in my favourite editor (Vim) and trying new tools.}
\end{document}
